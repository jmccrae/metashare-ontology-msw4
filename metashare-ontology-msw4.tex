\documentclass{llncs}
\usepackage{color}

\begin{document}

\title{Metashare as an ontology for the interoperability of linguistic datasets}

%
\titlerunning{Metashare ontology}  % abbreviated title (for running head)
%                                     also used for the TOC unless
%                                     \toctitle is used
%

% This is not the final order!
\author{Philipp Cimiano\inst{1} \and Jorge Gracia\inst{2} \and Penny Labropoulou\inst{3} \and John P. McCrae\inst{1} \and V\'ictor Rodr\'iguez Doncel\inst{2} \and Marta Villegas\inst{4}}
%
\authorrunning{Cimiano et al.} % abbreviated author list (for running head)
%
%
\institute{Cognitive Interaction Technology, Excellence Cluster, Bielefeld University, Inspiration 1, D-33619 Germany, \\
    \email{\{cimiano, jmccrae\}@cit-ec.uni-bielefeld.de}
\and
    Ontology Engineering Group, Universidad Polit\'ecnica de Madrid, Boadilla del Monte, Madrid, Spain \\
    \email{\{jgracia, vrodriguez\}@fi.upm.es}
\and
    Athena R.C./ILSP, Athens, Greece, \\
    \email{penny@ilsp.athena-innovation.gr}
\and
    University Pompeu Fabra, Barcelona, Spain, \\
\email{marta.villegas@upf.edu}}
    
\maketitle              % typeset the title of the contribution

\begin{abstract}
    \keywords{keywords}
\end{abstract}

\section{Introduction}

\cite{piperidis2012meta}

\section{Related Work}
\section{The META-SHARE Ontology}
\subsection{Original MS XSD schema [PL]}
\subsection{Purpose of the ontology [MV,JG,JPM]}
(e.g., why do RDF and OWL for an already defined vocabulary?) 
\subsection{Formal modelling and mapping issues [MV, JPM, PL]}
\subsection{Interface with DCAT and other vocabularies [JPM]}

\subsection{Licensing module [VRD, PL]}
\textcolor{red}{Victor: I suggest the following structure:}

FIRST: We describe first the Metashare schema, whose licensing information is described in an independent XML Schema file, available on git \footnote{\url{https://github.com/metashare/META-SHARE/blob/master/misc/schema/v3.0/META-SHARE-LicenseMetadata.xsd}}. 

SECOND: We then describe Marta's effort, and the META-SHARE ontology which is also publicly available\footnote{\url{https://raw.githubusercontent.com/martavillegas/metadata/master/MetaShare.ttl}}.

THIRD: We discuss on the needs that motivated the evolution from the previous model. 

FOURTH: We list the changes that we have introduced.


\section{META-SHARE in LingHub}
\subsection{LIXR mapping methodology [JPM]}
\subsection{Harmonizing MetaShare with other metadata sources [JPM]}
\section{Discussion}
\subsection{Applications of the MetaShare model (beyond LingHub) [MV]}
\subsection{5.2 Challenges and future outlooks [PC]}
\section{Conclusion}

\bibliographystyle{splncs03}
\bibliography{metashare-ontology-msw4}

\end{document}
