\documentclass{llncs}
\usepackage{color}
\usepackage{array}

\begin{document}

\title{Metashare as an ontology for the interoperability of linguistic datasets}

%
\titlerunning{Metashare ontology}  % abbreviated title (for running head)
%                                     also used for the TOC unless
%                                     \toctitle is used
%

% This is not the final order!
\author{Philipp Cimiano\inst{1} \and Jorge Gracia\inst{2} \and Penny Labropoulou\inst{3} \and John P. McCrae\inst{1} \and V\'ictor Rodr\'iguez Doncel\inst{2} \and Marta Villegas\inst{4}}
%
\authorrunning{Cimiano et al.} % abbreviated author list (for running head)
%
%
\institute{Cognitive Interaction Technology, Excellence Cluster, Bielefeld
    University, Inspiration 1, D-33619 Bielefeld, Germany, \\
    \email{\{cimiano, jmccrae\}@cit-ec.uni-bielefeld.de}
\and
    Ontology Engineering Group, Universidad Polit\'ecnica de Madrid, Boadilla del Monte, Madrid, Spain \\
    \email{\{jgracia, vrodriguez\}@fi.upm.es}
\and
    ILSP/Athena R.C., Athens, Greece, \\
    \email{penny@ilsp.athena-innovation.gr}
\and
    University Pompeu Fabra, Barcelona, Spain, \\
\email{marta.villegas@upf.edu}}
    
\maketitle              % typeset the title of the contribution

\begin{abstract}
    \keywords{keywords}
\end{abstract}

\section{Introduction}
\label{sec:introduction}

META-SHARE~\cite{piperidis2012meta} is a subproject of META-NET aiming to create high-quality metadata
for a large number of langauge resources and make them available in a structured
form. Up until now the main methodology that META-SHARE has used to make data
available is by means of XML Schema Definitions (XSD) and XML description of
individual language resources conforming to these definitions. However,
META-SHARE still covers only a small percentage of language resources and its
resource-intensive curation methodology means it is unlikely to cover all
language resources. In contrast, there have been a number of schemes that have
attempted to collect much more metadata from either existing institutional
repositories~\cite[CLARIN]{broeder2012data} or by crowd-sourcing this data from
researchers~\cite[LRE-Map]{calzolari2012lre}. The former method has lead to
data in different incompatible formats and the latter to noisy, incomplete and
duplicative records.

In this paper, we propose a solution to these issues by means of the development
of a single ontology for the representation of language resources based on the
original META-SHARE schema

\cite{piperidis2012meta}

\section{Related Work}
\label{sec:relatedwork}

\section{The META-SHARE Ontology}
\label{sec:ontology}

\subsection{Original MS XSD schema [PL]}
\label{sec:xsd}

\subsection{Purpose of the ontology [MV,JG,JPM]}
\label{sec:purpose}
(e.g., why do RDF and OWL for an already defined vocabulary?) 
\subsection{Formal modelling and mapping issues [MV, JPM, PL]}

When mapping an XML scheme to RDF there are naturally differences that must be
accounted for, which generic mapping methodologies cannot accommodate without
tending to a high degree of verbosity. The basic mapping strategy is ... [MV]. 

Beyond this, we made the following extensions to our mapping strategy:

\begin{itemize}
    \item We shorten some names such as ConformanceToBestStandardsAndPractices
        \textcolor{red}{JPM: Perhaps we introduce sameAs links to handle this}
    \item Developed novel classes based on existing values, e.g.,
        $\mathrm{Corpus} \equiv \exists \mathrm{resourceType}.\mathrm{corpus}$
    \item Removing unnecessary properties such as {\tt versionInfo}.
    \item Generalized elements such as {\tt userNature}, {\tt
        notAvailableThroughMetashare}
    \item Grouping similar elements under novel superclasses, e.g., {\tt
        DiscourseAnnotation}, {\tt genre}
    \item Extending existing classes with new values and including new
        properties (see section \ref{sec:licensing}
\end{itemize}

\subsection{Interface with DCAT and other vocabularies [JPM]}
\label{sec:dcat}

\subsection{Licensing module [VRD, PL]}
\label{sec:licensing}
\textcolor{red}{Skeleton}

We describe first the Metashare schema, whose licensing information is described in an independent XML Schema file, available on git \footnote{\url{https://github.com/metashare/META-SHARE/blob/master/misc/schema/v3.0/META-SHARE-LicenseMetadata.xsd}}. 

%SECOND: We then describe Marta's effort, and the META-SHARE ontology which is also publicly available\footnote{\url{https://raw.githubusercontent.com/martavillegas/metadata/master/MetaShare.ttl}}.

We discuss on the needs that motivated the evolution from the previous model. We describe (if not done before) also the precedure and methodology.

Short introduction on the ODRL vocabulary. 

We describe the most important changes that we have introduced.

And going beyond ODRL: License Templates as an easy entry points for Semantic Web - laymans.

Example of license template, example of license. Directly in TTL. Maybe introducing a figure depecting what is metadata for resource/distrubiont/license? 

\section{META-SHARE in LingHub}
\label{sec:linghub}

LingHub~\footnote{\url{http://linghub.org}} is a large resource containing information about a wide range of
language resources, but unlike META-SHARE it does not directly collect this
information, but instead harmonizes the metadata from a wide range of sources.
In this section, we will first describe how the original META-SHARE data was
translated into RDF and the alignment with DCAT~\cite{maali2014data}, previously
described, was achieved. Furthermore, we will then consider how we have used the
META-SHARE vocabulary as a base vocabulary to align terms from other resources
included in LingHub.

\subsection{Mapping META-SHARE to RDF [JPM]}
\label{sec:conversion}

When translating XML documents into RDF, one of the most common approaches is
based on Extensible Stylesheet Language Transformations
(XSLT)~\cite{wustner2002converting,van2008xml,borin2014representing}, which has
been extended by some authors into a significant
framework~\cite{lange2009krextor}. However, XSLT has a number of disadvantages
for this task:

\begin{itemize}
    \item The set of functions and operators suported by most processors is
        limited.
    \item Limited ability to declare new functions.
    \item Does not allow stream (SAX) processing of large files.
    \item XSLT is a one-way transformation language and it is not this possible
        to `round-trip' the conversion, i.e., convert RDF to XML.
    \item XSLT syntax is expressed in XML and thus is very verbose and
        aesthetically unpleasing. For this reason, many people use alternative
        more compact syntaxes\footnote{Compact XML:
        \url{https://pythonhosted.org/compactxml/}}\footnote{Jade:
        \url{http://jade-lang.com/}}
\end{itemize}

Furthermore, the META-SHARE syntax is very complex consisting of 111 complex
types and 207 simple types. As such we deemed that the development of a new
language for transformation and writing our converter in that language would
take less development effort than writing a conversion entirely in XSLT. The
mapping methodology we developed is a domain-specific
langauge~\cite{fowler2010domain} called Lightweight Invertible XML to RDF
Conversion (LIXR) and aims to improve on the situation by fixing the concerns
above. 

To begin with we selected the Scala programming language as the basis for LIXR
as it has a proven syntactic flexibility that makes it easy to write
domain-specific languages~\cite{wampler2008programming}. A simple example of a
LIXR mapping is given below:

{\footnotesize
\begin{verbatim}
object Metashare extends eu.liderproject.lixr.Model {
  val dc = Namespace("http://purl.org/dc/elements/1.1/")
  val ms = Namespace("http://purl.org/ms-lod/MetaShare.ttl#")
  val msxml = Namespace("http://www.ilsp.gr/META-XMLSchema")

  msxml.resourceInfo --> (
    a > ms.ResourceInfo,
    handle(msxml.identificationInfo)
  )

  msxml.identificationInfo --> (
    dc.title > (content(msxml.resourceName) @@ msxml.resourceName.att("lang"))
  )
}
\end{verbatim}}

In this example, we first create our model extending the basic LIXR model and
define namespaces as dynamic Scala objects\footnote{This is a newer feature of
    Scala only supported since 2.10 (Jan 2013)}. We then make two mapping
declarations for the tags {\tt resourceInfo} and {\tt identificationInfo}. LIXR (as 
XSLT) simply searches for a matching declaration at the root of the XML document
to begin the transformation. Having matched the {\tt resourceInfo} tag, the system
first generates the triple that states that the base element has type
{\tt ms:resourceInfo}, and then `handles' any children {\tt identificationInfo} tags by
searching for an appropriate rule for each one. For {\tt identificationInfo} the
system generates a triple using the {\tt dc:title} property whose value is the
content of the {\tt resourceName} tag tagged with the language given by the
attribute {\tt lang}.

\begin{table}
    \begin{center}
    \begin{tabular}{p{4cm}|cccc}
        Name        & Tags      & Implementation & LoC    & LoC/Tag \\
        \hline
        TBX         & 48        & Java           & 2,752  & 57.33   \\
        CLARIN (OLAC-DMCI) & 79 & XSLT           & 404    & 5.11    \\
        CLARIN (OLAC-DMCI) & 79 & XSLT (Compact Syntax) & 255    & 3.22    \\
        \hline
        TBX         & 48        & LIXR           & 197    & 4.10    \\
        CLARIN (OLAC-DMCI) & 79 & LIXR           & 176    & 2.23    \\
        MetaShare   & 730       & LIXR           & 2,487  & 3.41    \\
    \end{tabular}
\end{center}
    \caption{\label{tab:locs}Comparison of XML to RDF mapping implementations,
    by number of tags in XML schema, and non-trivial lines of code (LoC)}
\end{table}

To evaluate the effectiveness of our approach we compared directly with two other
XML to RDF transformations, we had carried out in this project, and
reimplemented them using the LIXR language. In particular these were the TBX
model~\cite{iso30042} as well as the OLCA-DMCI profile of the CLARIN
metadata~\footnote{\url{http://catalog.clarin.eu/ds/ComponentRegistry/rest/registry/profiles/clarin.eu:cr1:p\_1288172614026/xsd}}. In table \ref{tab:locs}, we see the
effort to implement these using LIXR is approximately half of using XSLT and
about ten times less than writing a converter from scratch. 

In addition to the reduction in effort using this approach, we also note several
other advantages of the LIXR approach, due to its declarative declaration

\begin{itemize}
    \item We can easily switch to using a stream-based parse for XML (e.g., SAX)
        so we can process large files without having to use much memory
    \item A reverse mapping can be extracted that re-generates the XML from the
        outputted RDF
    \item We can extract the type, range and domain of RDF entities generated
        during this procedure. This export formed the initial version of the
        ontology described in this paper
\end{itemize}

\subsection{Harmonizing other resources with META-SHARE [JPM]}
\label{sec:harmonization}

LingHub brings resources from a wide-range of sources and while we can use
standards such as DCAT and Dublin Core to guarantee a common representation of the basic
Metadata of a resource, there does not a standard for the represenation of
metadata specific to linguistics. For that reason, we use the META-SHARE model
as a standard for other resources to use in LingHub. In particular, we will
focus on the application of META-SHARE as a model to harmonize CLARIN data. The
CLARIN repository describes its resources using a small common set of metadata
and a larger description defined by the Component Metadata
Infrastructure~\cite[CMDI]{broeder2012cmdi}. These metadata schemes are
extremely diverse as shown in table \ref{tab:clarin-types}.

\begin{table}
    \begin{tabular}{l|lc}
        Component Root Tag & Institutes & Frequency \\
        \hline
        Song               & 1 (MI)     & 155,403   \\
        Session            & 1 (MPI)    & 128,673   \\
        OLAC-DcmiTerms     & 39         &  95,370   \\
        mods               & 1 (Utrecht)&  64,632   \\
        DcmiTerms          & 2 (BeG,HI) &  46,160   \\
        SongScan           & 1 (MI)     &  28,448   \\
        media-session-profile  & 1 (Munich) &  22,405   \\
        SourceScan         & 1 (MI)     &  21,256   \\
        Source             & 1 (MI)     &  16,519   \\
        teiHeader          & 2 (BBAW, Copenhagen) &  15,998   \\
    \end{tabular}
    \caption{\label{tab:clarin-types}The top 10 most frequent component types in
    CLARIN and the institutes that use them. Abbreviations: MI=Meertens Institute (KNAW), 
    MPI=Max Planck Insitute (Nijmegen), BeG=Netherlands Institute for Sound and Vision,
HI=Huygens Institute (KNAW), BBAW=Berlin-Brandenburg Academy of Sciences}
\end{table}

\section{Discussion}
\label{sec:discussion}

\subsection{Applications of the MetaShare model (beyond LingHub) [MV]}
\label{sec:applications}

\subsection{Challenges and future outlooks [PC]}
\label{sec:challenges}

\section{Conclusion}
\label{sec:conclusion}

\bibliographystyle{splncs03}
\bibliography{metashare-ontology-msw4}

\end{document}
